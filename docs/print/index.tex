% Author: Daniel Vartanian.
% Licence: MIT. See <https://opensource.org/license/mit/> to learn more.
%
% Based on: template.tex, developed by the Quarto team and
%   abtex2-modelo-trabalho-academico.tex, v-1.9.7, developed by
%   Lauro César Araujo and the team behind abnt2tex, with additional guidance
%   from the theses and dissertations regulations of the University of São Paulo
%   (USP). For more information, please visit <http://www.abntex.net.br/>.

% For help, see:
%
% - <https://quarto.org/docs/reference/formats/pdf.html>
% - <https://github.com/abntex/abntex2/wiki/ComoCustomizar>
% - <https://www.ctan.org/pkg/abntex2>
% - <https://www.ctan.org/pkg/memoir>
% - <https://www.ctan.org/pkg/hyperref>

% TO DO:
%
% * Slightly move the toc to the left, in a way that the spacing between titles
%   and numbers become the same as the textual chapters.
% * Remove hiperlink spans by page breaks. See: <https://tex.stackexchange.com/questions/54136/hyperref-link-spans-a-pagebreak-looks-ugly>.

% -----
% Preamble
% -----

% Don´t move the `\PassOptionsToPackage` macros.
% See why: <https://tex.stackexchange.com/a/433605/234832>.
\PassOptionsToPackage{
unicode,bookmarksnumbered
}{hyperref}

\PassOptionsToPackage{hyphens}{url}

\PassOptionsToPackage{dvipsnames,svgnames,x11names}{xcolor}


\documentclass[
12pt,
openright,
oneside,
a4paper,
chapter=TITLE,
section=TITLE,
french,
spanish,
brazil,
english
]{abntex2}
% \usepackage{showframe}

% The order in which the packages are loaded is important!

\usepackage{array}
\usepackage{calc}
\usepackage{caption}
\usepackage{color}
\usepackage{colortbl}
\usepackage{amsmath}
\usepackage{amssymb}
\usepackage{booktabs}
\usepackage{enumitem}
\usepackage{etoolbox}
\usepackage{epigraph}
\usepackage{float}
\usepackage[T1]{fontenc}
\usepackage[hang,multiple]{footmisc}
\usepackage{graphicx}
\usepackage{iftex}
\usepackage{indentfirst}
\usepackage[hyphenation,lastparline,nosingleletter]{impnattypo}
\usepackage[utf8]{inputenc}
\usepackage{lastpage}
\usepackage{lipsum}
\usepackage{longtable}
\usepackage{luacode}
\usepackage{luatexbase}
\usepackage{microtype}
\usepackage{multirow}
\usepackage{parskip}
\usepackage{pdfpages}
\usepackage{titlesec}
\usepackage[table]{xcolor}
\usepackage{xparse}
\usepackage{xstring}
\usepackage{hyperref}
\usepackage{url}

\ifPDFTeX
  \usepackage{textcomp} % provide euro and other symbols
\else % if luatex or xetex
\usepackage{unicode-math}
\fi
% Set lengths -----

\newlength{\microskipamount}
\newlength{\tinyskipamount}
\newlength{\hugeskipamount}

\setlength{\microskipamount}{0.25\baselineskip} % Arial/12pt/1.5 == 5.4375pt
\setlength{\tinyskipamount}{0.5\baselineskip} % Arial/12pt/1.5 == 10.875pt
\setlength{\smallskipamount}{0.75\baselineskip} % Arial/12pt/1.5 == 16.3125pt
\setlength{\medskipamount}{1\baselineskip} % Arial/12pt/1.5 == 21.75pt
\setlength{\bigskipamount}{1.5\baselineskip} % Arial/12pt/1.5 == 32.625pt
\setlength{\hugeskipamount}{2\baselineskip}% Arial/12pt/1.5 == 43.5pt

\newcommand{\microskip}{\vspace{\microskipamount}}
\newcommand{\tinyskip}{\vspace{\tinyskipamount}}
\newcommand{\hugeskip}{\vspace{\hugeskipamount}}

% Set skips -----

\setlength{\beforechapskip}{\bigskipamount}
\setlength{\afterchapskip}{\smallskipamount}

\titlespacing*{\chapter}{0pt}{\beforechapskip}{\afterchapskip}
\titlespacing*{\section}{0pt}{\medskipamount}{\smallskipamount}
\titlespacing*{\subsection}{0pt}{\medskipamount}{\smallskipamount}
\titlespacing*{\subsubsection}{0pt}{\medskipamount}{\smallskipamount}
\titlespacing*{\paragraph}{0pt}{\medskipamount}{\smallskipamount}

% Set epigraph -----

\setlength\epigraphwidth{0.6\textwidth}
\setlength\epigraphrule{0pt}
% Set page -----

\setlength{\headsep}{1cm}
\setlength{\footskip}{1cm}
\checkandfixthelayout[fixed]

% Set text spacing -----

\renewcommand{\familydefault}{\rmdefault}

\renewcommand{\baselinestretch}{1.5}

\setlength{\parindent}{1cm}

\setlength{\parskip}{0ex}

% Set text font -----


\ifPDFTeX\else
    % xetex/luatex font selection
  \setmainfont[]{Open Sans}
  \setsansfont[]{Open Sans}
  \setmonofont[Scale=0.75]{Source Code Pro}




\fi


% Set footnote -----

\setlength{\footnotemargin}{0.5em} % Equal to `\footmarkwidth`
\let\svfootnoterule\footnoterule % Equal to `\footmarksep`
\renewcommand\footnoterule{\vspace{1ex}\svfootnoterule\vspace{1ex}}
\newcommand{\capaname}{Capa}
\newcommand{\fichacatalograficaname}{Ficha catalográfica}
\newcommand{\resumoestrangeironame}{Resumo}
\newcommand{\glossarioname}{Glossário}

\addto\captionsenglish{
  \renewcommand{\capaname}{Cover}
  \renewcommand{\folhaderostoname}{Title Page}
  \renewcommand{\fichacatalograficaname}{Cataloging Record}
  \renewcommand{\errataname}{Errata}
  \renewcommand{\folhadeaprovacaoname}{Approval Sheet}
  \renewcommand{\dedicatorianame}{Inscription}
  \renewcommand{\agradecimentosname}{Acknowledgements}
  \renewcommand{\epigraphname}{Epigraph}
  \renewcommand{\resumoname}{Abstract}
  \renewcommand{\resumoestrangeironame}{Resumo}
  \renewcommand{\listfigurename}{List of Figures}
  \renewcommand{\listtablename}{List of Tables}
  \renewcommand{\listadesiglasname}{List of Abbreviations and Acronyms}
  \renewcommand{\listadesimbolosname}{List of Symbols}
  \renewcommand{\contentsname}{Contents}
  \renewcommand{\bibname}{References}
  \renewcommand{\glossarioname}{Glossary}
  \renewcommand{\apendicename}{APPENDIX}
  \renewcommand{\apendicesname}{Appendices}
  \renewcommand{\anexoname}{ANNEX}
  \renewcommand{\anexosname}{Annexes}
  \renewcommand{\indexname}{Index}
  \renewcommand{\orientadorname}{Supervisor:}
  \renewcommand{\coorientadorname}{Co-supervisor:}
  \renewcommand{\fontename}{Source}
  \renewcommand{\notaname}{Note}
  \renewcommand{\pageautorefname}{page}
  \renewcommand{\sectionautorefname}{section}
  \renewcommand{\subsectionautorefname}{subsection}
  \renewcommand{\subsubsectionautorefname}{subsubsection}
  \renewcommand{\paragraphautorefname}{subsubsubsection}
}

\addto\captionsbrazil{
  \renewcommand{\capaname}{Capa}
  \renewcommand{\folhaderostoname}{Folha de Rosto}
  \renewcommand{\fichacatalograficaname}{Ficha Catalográfica}
  \renewcommand{\errataname}{Errata}
  \renewcommand{\folhadeaprovacaoname}{Folha de Aprovação}
  \renewcommand{\dedicatorianame}{Dedicatória}
  \renewcommand{\agradecimentosname}{Agradecimentos}
  \renewcommand{\epigraphname}{Epígrafe}
  \renewcommand{\resumoname}{Resumo}
  \renewcommand{\resumoestrangeironame}{Abstract}
  \renewcommand{\listfigurename}{Lista de Figuras}
  \renewcommand{\listtablename}{Lista de Tabelas}
  \renewcommand{\listadesiglasname}{Lista de Abreviaturas e Siglas}
  \renewcommand{\listadesimbolosname}{Lista de Símbolos}
  \renewcommand{\contentsname}{Sumário}
  \renewcommand{\bibname}{Referências}
  \renewcommand{\glossarioname}{Glossário}
  \renewcommand{\apendicename}{APÊNDICE}
  \renewcommand{\apendicesname}{Apêndices}
  \renewcommand{\anexoname}{ANEXO}
  \renewcommand{\anexosname}{Anexos}
  \renewcommand{\indexname}{Índice}
  \renewcommand{\orientadorname}{Orientador:}
  \renewcommand{\coorientadorname}{Coorientador:}
  \renewcommand{\fontename}{Fonte}
  \renewcommand{\notaname}{Nota}
  \renewcommand{\pageautorefname}{página}
  \renewcommand{\sectionautorefname}{seção}
  \renewcommand{\subsectionautorefname}{subseção}
  \renewcommand{\subsubsectionautorefname}{subsubseção}
  \renewcommand{\paragraphautorefname}{subsubsubseção}
}

\addto\captionsspanish{
  \renewcommand{\capaname}{Portada}
  \renewcommand{\folhaderostoname}{Página de título}
  \renewcommand{\fichacatalograficaname}{Ficha Catalográfica}
  \renewcommand{\errataname}{Errata}
  \renewcommand{\folhadeaprovacaoname}{Hoja de aprobación}
  \renewcommand{\dedicatorianame}{Dedicatoria}
  \renewcommand{\agradecimentosname}{Agradecimientos}
  \renewcommand{\epigraphname}{Epígrafe}
  \renewcommand{\resumoname}{Resumen}
  \renewcommand{\resumoestrangeironame}{Resumo}
  \renewcommand{\listfigurename}{Lista de Figuras}
  \renewcommand{\listtablename}{Lista de Tablas}
  \renewcommand{\listadesiglasname}{Lista de Abreviaturas y Siglas}
  \renewcommand{\listadesimbolosname}{Lista de Símbolos}
  \renewcommand{\contentsname}{Sumario}
  \renewcommand{\bibname}{Referencias}
  \renewcommand{\glossarioname}{Glosario}
  \renewcommand{\apendicename}{APÉNDICE}
  \renewcommand{\apendicesname}{Apéndices}
  \renewcommand{\anexoname}{ANEXO}
  \renewcommand{\anexosname}{Anexos}
  \renewcommand{\indexname}{Índice}
  \renewcommand{\orientadorname}{Asesor:}
  \renewcommand{\coorientadorname}{Coasesor:}
  \renewcommand{\fontename}{Fuente}
  \renewcommand{\notaname}{Nota}
  \renewcommand{\pageautorefname}{página}
  \renewcommand{\sectionautorefname}{sección}
  \renewcommand{\subsectionautorefname}{subsección}
  \renewcommand{\subsubsectionautorefname}{subsubsección}
  \renewcommand{\paragraphautorefname}{subsubsubsección}
}

\addto\captionsfrench{
  \renewcommand{\capaname}{Couverture}
  \renewcommand{\folhaderostoname}{Page de Titre}
  \renewcommand{\fichacatalograficaname}{Fiche Cataloguée}
  \renewcommand{\errataname}{Errata}
  \renewcommand{\folhadeaprovacaoname}{Feuille d'Approbation}
  \renewcommand{\dedicatorianame}{Dédiace
  \renewcommand{\agradecimentosname}{Remerciements}}
  \renewcommand{\epigraphname}{Épigraphe}
  \renewcommand{\resumoname}{Résumé}
  \renewcommand{\resumoestrangeironame}{Resumo}
  \renewcommand{\listfigurename}{Liste des Figures}
  \renewcommand{\listtablename}{Liste des Tableaux}
  \renewcommand{\listadesiglasname}{Liste des Abréviations et Sigles}
  \renewcommand{\listadesimbolosname}{Liste des Symboles}
  \renewcommand{\contentsname}{Sommaire}
  \renewcommand{\bibname}{Références}
  \renewcommand{\glossarioname}{Glossaire}
  \renewcommand{\apendicename}{APPENDICE}
  \renewcommand{\apendicesname}{Appendices}
  \renewcommand{\anexoname}{ANNEXE}
  \renewcommand{\anexosname}{Annexes}
  \renewcommand{\indexname}{Index}
  \renewcommand{\orientadorname}{Conseiller:}
  \renewcommand{\coorientadorname}{Co-conseiller:}
  \renewcommand{\fontename}{Source}
  \renewcommand{\notaname}{Note}
  \renewcommand{\pageautorefname}{page}
  \renewcommand{\sectionautorefname}{section}
  \renewcommand{\subsectionautorefname}{sous-section}
  \renewcommand{\subsubsectionautorefname}{sous-sous-section}
  \renewcommand{\paragraphautorefname}{sous-sous-sous-section}
}
% See `babel.tex` for language changes.
% See `toc.text` for changes related to the ToC.

% Set page numbering -----

\makepagestyle{abntheadings}
\makeevenhead{abntheadings}{\ABNTEXfontereduzida\thepage}{}{}
\makeoddhead{abntheadings}{}{}{\ABNTEXfontereduzida\thepage}

% Set text variables -----

\renewcommand{\ABNTEXpartfont}{\sffamily\bfseries}
\renewcommand{\ABNTEXpartfontsize}{\normalsize}
\renewcommand{\ABNTEXchapterfont}{\sffamily\bfseries}
\renewcommand{\ABNTEXchapterfontsize}{\normalsize}
\renewcommand{\ABNTEXsectionfont}{\sffamily}
\renewcommand{\ABNTEXsectionfontsize}{\normalsize}
\renewcommand{\ABNTEXsubsectionfont}{\sffamily}
\renewcommand{\ABNTEXsubsectionfontsize}{\normalsize}
\renewcommand{\ABNTEXsubsubsectionfont}{\sffamily}
\renewcommand{\ABNTEXsubsubsectionfontsize}{\normalsize}
\renewcommand{\ABNTEXsubsubsubsectionfont}{\sffamily}
\renewcommand{\ABNTEXsubsubsubsectionfontsize}{\normalsize\itshape}
\renewcommand{\ABNTEXfontereduzida}{\footnotesize}
\renewcommand{\ABNTEXcaptiondelim}{~\textendash~}
\renewcommand{\ABNTEXcaptionfontedelim}{:~}

\renewcommand{\captiontitlefont}{\ABNTEXfontereduzida}

% Set new commands -----

\providecommand{\imprimiruniversidade}{}
\newcommand{\universidade}[1]{\renewcommand{\imprimiruniversidade}{#1}}

\providecommand{\imprimirescola}{}
\newcommand{\escola}[1]{\renewcommand{\imprimirescola}{#1}}

\providecommand{\imprimirprograma}{}
\newcommand{\programa}[1]{\renewcommand{\imprimirprograma}{#1}}

\newcommand{\imprimirtipodetrabalho}{\imprimirtipotrabalho}

\providecommand{\imprimirtipodetituloacademico}{}
\newcommand{\tipodetituloacademico}[1]{\renewcommand{\imprimirtipodetituloacademico}{#1}}

\providecommand{\imprimirtituloacademico}{}
\newcommand{\tituloacademico}[1]{\renewcommand{\imprimirtituloacademico}{#1}}

\providecommand{\imprimirareadeconcentracao}{}
\newcommand{\areadeconcentracao}[1]{\renewcommand{\imprimirareadeconcentracao}{#1}}

\providecommand{\imprimirnotadeversao}{}
\newcommand{\notadeversao}[1]{\renewcommand{\imprimirnotadeversao}{#1}}

% Set chapter style -----

\renewcommand{\chapnamefont}{\ABNTEXchapterfont\ABNTEXchapterfontsize\mdseries}
\renewcommand{\chapnumfont}{\ABNTEXchapterfont\ABNTEXchapterfontsize\mdseries}

\setsecnumformat{\chapnumfont\csname the#1\endcsname\quad}

\renewcommand{\printchaptername}{
  \ifthenelse{\boolean{abntex@apendiceousecao}}{
    \vspace*{\medskipamount}
    \chapnamefont \ABNTEXchapterupperifneeded{\appendixname} % [Changed]
  }{}
}

% Open an issue about it (`\hspace{-1em}`) - Title stretching.
\renewcommand{\chapternamenum}{
  \ifthenelse{\boolean{abntex@apendiceousecao}}{
    \hspace{-2em} \space
  }{}
}

\renewcommand{\printchapternum}{
  \tocprintchapter
  \setboolean{abntex@innonumchapter}{false}
  \chapnumfont
  \thechapter % [Changed]
  % \ifthenelse{\boolean{abntex@apendiceousecao}}{ % [Removed]
  %   \tocinnonumchapter
  %   \ABNTEXcaptiondelim
  % }{}
}

\renewcommand{\afterchapternum}{
  \ifthenelse{\boolean{abntex@apendiceousecao}}{ % [Added]
    \ABNTEXchapterfont\mdseries \hspace{-1em} \space\ABNTEXcaptiondelim\space \hspace{-1.5em}
  }{
    \hspace{-0.875em}
  }
}

\renewcommand{\printchapternonum}{
  \tocprintchapternonum
  \setlength{\afterchapskip}{\hugeskipamount} % [Added]
  \setboolean{abntex@innonumchapter}{true}
}

\renewcommand{\printchaptertitle}[1]{
  \chaptitlefont
  \ifthenelse{\boolean{abntex@innonumchapter}}{
    \centering \ABNTEXchapterupperifneeded{#1}
  }{
    \ifthenelse{\boolean{abntex@apendiceousecao}}{
      \ABNTEXchapterfont\mdseries\ABNTEXchapterupperifneeded{#1}
    }{
      \ABNTEXchapterupperifneeded{#1}
    }
  }
}

% Set `\textual` -----

\renewcommand{\textual}{
  \pagestyle{abntheadings}
  \aliaspagestyle{chapter}{abntheadings}
}

% Set cover -----

\renewcommand{\imprimircapa}{
  \phantomsection\pdfbookmark[0]{\capaname}{}
  \begin{capa}%
  \begin{adjustwidth}{-1cm}{0cm}
  \center
  \imprimirinstituicao

  \vfill
  \imprimirautor

  \vfill
  {\ABNTEXchapterfont\imprimirtitulo}

  \vfill
  \vspace{6.5cm}
  \imprimirlocal

  \imprimirdata
  \vspace{1.5cm}
  \end{adjustwidth}
  \end{capa}
}

% Set title page -----

\makeatletter
\renewcommand{\folhaderostocontent}{
  \begin{center}
  \imprimirautor

  \vfill
  {\ABNTEXchapterfont\imprimirtitulo}

  \vfill
  \textbf{\imprimirnotadeversao}

  \vfill
  \abntex@ifnotempty{
    \imprimirpreambulo
  }{
    \hspace{0.35\textwidth}
    \begin{minipage}{.6\textwidth}
    \SingleSpacing
    \imprimirpreambulo
    \end{minipage}
  }

  \vfill
  \imprimirlocal

  \imprimirdata
  \vspace{1cm}
  \end{center}
}
\makeatother

% Set cataloging record -----

\renewenvironment{fichacatalografica}{
  \PRIVATEbookmarkthis{\fichacatalograficaname}
  \setlength{\parindent}{0cm}
  \begin{SingleSpacing}
}{
  \end{SingleSpacing}
}

% Set errata -----

\renewenvironment{errata}[1][\errataname]{
  \newpage
  \phantomsection
  \pretextualchapter{#1}
}{
  \cleardoublepage
}

% Set approval sheet -----

\renewenvironment{folhadeaprovacao}[1][\folhadeaprovacaoname]{
  \clearpage
  \PRIVATEbookmarkthis{#1}
  \setlength\parindent{0cm}
  \AtBeginEnvironment{tabular}{\normalsize}
  \begin{SingleSpace}
}{
  \end{SingleSpace}
  \cleardoublepage
}

% Set abstract -----

\newenvironment{resumoenv}[1][\resumoname]{
  \pretextualchapter{#1}
  \begingroup
  \setlength{\parindent}{0cm}
  \setlength{\parskip}{\smallskipamount} % The troublemaker.
  \AtBeginEnvironment{tabular}{\normalsize}
  \renewcommand{\arraystretch}{1}
  \setlength{\aboverulesep}{0ex}
  \setlength{\belowrulesep}{0ex}
  \setlength{\arrayrulewidth}{0pt}
  \setlength{\tabcolsep}{0cm}
  \vspace{-\smallskipamount} % !
  \begin{SingleSpace}
}{
  \end{SingleSpace}
  \cleardoublepage
  \endgroup
}

% Set list of abbreviations and acronyms -----

\renewenvironment{siglas}{
  \pretextualchapter{\listadesiglasname}
}{
  \cleardoublepage
}

% Set list of symbols -----

\renewenvironment{simbolos}{
  \pretextualchapter{\listadesimbolosname}
}{
  \cleardoublepage
}

% Set glossary -----

\newenvironment{glossario}{
  \tocprintchapternonum
}{
  \cleardoublepage
}

% Set appendices and annexes -----

\renewcommand{\PRIVATEapendiceconfig}[2]{
  \setboolean{abntex@apendiceousecao}{true}
  \renewcommand{\appendixname}{#1}
  %\renewcommand{\apendicesname}{#1}

  \ifthenelse{\boolean{ABNTEXsumario-abnt-6027-2012}}{
    \renewcommand{\appendixtocname}{\uppercase{#2}}
  }{
    \renewcommand{\appendixtocname}{#2}
  }

  \renewcommand{\appendixpagename}{#2}
  \renewcommand{\appendixtocname}{#2}
  % \switchchapname{#1} % [Altered]
  \renewcommand{\cftappendixname}{} % [Altered]
  \tocpartapendices % [Added]

  % Note:
  %
  % \cleardoublepage
  % \phantomsection
  % \addcontentsline{toc}{part}{Appendices}
  % \appendix
  %
  % is automatically add by the Quarto render.
}

\newcommand{\PRIVATEapendiceconfigafter}[1]{
    \chapterstyle{apendice}
    %\begingroup\centering\bfseries
    %\ABNTEXchapterupperifneeded{#1}
    %\par\endgroup
    %\vspace{\medskipamount}
    \pretextualchapter{#1}
    \let\clearpage\relax
}

\renewcommand{\apendices}{
  \clearpage
  \PRIVATEapendiceconfig{\apendicename}{\apendicesname}
  \appendix
  \PRIVATEapendiceconfigafter{\apendicesname}
}

\renewenvironment{apendicesenv}{
  \clearpage
  \PRIVATEapendiceconfig{\apendicename}{\apendicesname}
  \begin{appendix}
  \PRIVATEapendiceconfigafter{\apendicesname}
}{
  \end{appendix}
  \setboolean{abntex@apendiceousecao}{false}
  \bookmarksetup{startatroot}
}

\renewcommand{\anexos}{
  \clearpage
  % \cftinserthook{toc}{AAA} [Removed]
  \PRIVATEapendiceconfig{\anexoname}{\anexosname}

  \newpage % [Added]
  \phantomsection % [Added]
  \addcontentsline{toc}{part}{\appendixtocname} % [Added]

  \appendix
  \renewcommand\theHchapter{anexochapback.\arabic{chapter}}
  \PRIVATEapendiceconfigafter{\anexosname}
}

\renewenvironment{anexosenv}{
  \clearpage
  \PRIVATEapendiceconfig{\anexoname}{\anexosname}

  \newpage % [Added]
  \phantomsection % [Added]
  \addcontentsline{toc}{part}{\appendixtocname} % [Added]

  \begin{appendix}
  \renewcommand\theHchapter{anexochapback.\arabic{chapter}}
  \PRIVATEapendiceconfigafter{\anexosname}
}{
  \end{appendix}
  \setboolean{abntex@apendiceousecao}{false}
  \bookmarksetup{startatroot}
}
% -----
% New colors
% -----

\definecolor{blue}{HTML}{2905C3}

% See <https://getbootstrap.com/docs/5.0/utilities/colors/>.
\definecolor{quarto-blue}{HTML}{2780E3}
\definecolor{quarto-lighter-blue}{HTML}{ECF4FC}
\definecolor{quarto-orange}{HTML}{FF7518}
\definecolor{quarto-ligther-orange}{HTML}{FFF3EB}
\definecolor{quarto-red}{HTML}{D9534F}
\definecolor{quarto-ligther-red}{HTML}{FCF1F1}
\definecolor{quarto-green}{HTML}{3FB618}
\definecolor{quarto-ligther-green}{HTML}{EFF9EB}
\definecolor{quarto-purple}{HTML}{7D12BA}
\definecolor{quarto-gray}{HTML}{A3A3A3}
\definecolor{quarto-medium-gray}{HTML}{CFD0D1}
\definecolor{quarto-ligther-gray}{HTML}{F1F3F5}

\definecolor{bs-link-color}{HTML}{39729E}

% -----
% Body color
% -----

\definecolor{body-color}{HTML}{63483E}
\color{body-color}

% Quarto's default settings -----

% \usepackage{graphicx} % Already loaded in `packages.tex`.
\makeatletter
\newsavebox\pandoc@box
\newcommand*\pandocbounded[1]{% scales image to fit in text height/width
  \sbox\pandoc@box{#1}%
  \Gscale@div\@tempa{\textheight}{\dimexpr\ht\pandoc@box+\dp\pandoc@box\relax}%
  \Gscale@div\@tempb{\linewidth}{\wd\pandoc@box}%
  \ifdim\@tempb\p@<\@tempa\p@\let\@tempa\@tempb\fi% select the smaller of both
  \ifdim\@tempa\p@<\p@\scalebox{\@tempa}{\usebox\pandoc@box}%
  \else\usebox{\pandoc@box}%
  \fi%
}
% Set default figure placement to htbp
\def\fps@figure{htbp}
\makeatother

% Set distance from top of page to first float -----

\makeatletter
\setlength{\@fptop}{5pt}
\makeatother

% Set captions and legends -----

\DeclareCaptionFont{ABNTEXfontereduzida}{\ABNTEXfontereduzida}

% For customization, see `\DeclareCaptionFormat` in the `caption` package.
\captionsetup{
  font=ABNTEXfontereduzida
  ,justification=justified
}

\renewcommand{\abovecaptionskip}{\smallskipamount}
\renewcommand{\belowcaptionskip}{\smallskipamount}

\renewcommand{\legend}[1]{
  \hyphenpenalty=100000
  \ABNTEXfontereduzida
  \addvspace{\smallskipamount}
  #1
}

% Credits: <https://tex.stackexchange.com/a/611556/234832>.
\AddToHook{cmd/caption/before}{\hyphenpenalty=100000}

% Set figure environment -----

\AtBeginEnvironment{figure}{
  \ABNTEXfontereduzida
  \addvspace{\tinyskipamount}
}

\AtEndEnvironment{figure}{
  \addvspace{\smallskipamount}
}
\renewcommand{\arraystretch}{1.5}
\setlength{\aboverulesep}{0ex}
\setlength{\belowrulesep}{0ex}

% Correct order of tables after \paragraph or \subparagraph
% \usepackage{etoolbox}
% \makeatletter
% \patchcmd\longtable{\par}{\if@noskipsec\mbox{}\fi\par}{}{}
% \makeatother

% Allow footnotes in `longtable` head/foot
\IfFileExists{footnotehyper.sty}{\usepackage{footnotehyper}}{\usepackage{footnote}}
\makesavenoteenv{longtable}

% Set tabular environment -----

\AtBeginEnvironment{table}{\ABNTEXfontereduzida}
\AtBeginEnvironment{tabular}{\ABNTEXfontereduzida}

\AtBeginEnvironment{longtable}{\ABNTEXfontereduzida \addvspace{\tinyskipamount}}
\AtBeginEnvironment{longtable*}{\ABNTEXfontereduzida \addvspace{\tinyskipamount}}

\floatplacement{table}{H}

% Set theorem environment -----

\AtEndEnvironment{theorem}{\vspace{\bigskipamount}}
\providecommand{\tightlist}{
\setlength{\itemsep}{0ex}\setlength{\parskip}{0\baselineskip}}

% \setlist[enumerate]{leftmargin=1cm)}
% \setlist[itemize]{leftmargin=2cm}
\makeatletter
\newcommand*{\getlength}[1]{\strip@pt#1}
\makeatother
\title{
\MakeTitlecase{Associations Between the Duration and Quality of Sleep of
Pregnant Women in the Third Trimester With the Duration of Labor}

}

\titulo{
\MakeTitlecase{Associations Between the Duration and Quality of Sleep of
Pregnant Women in the Third Trimester With the Duration of Labor}
}


\author{Alícia Rafaelly Vilefort Sales}
\autor{Alícia Rafaelly Vilefort Sales}

\local{São Paulo}

\date{2025}
\data{2025}

\orientador{Christiane Borges do Nascimento Chofakian}

\coorientador{{[}Co-supervisor's full name{]}}

\tipodetituloacademico{Master}

\tituloacademico{Master of Science}

\tipotrabalho{Thesis}

\areadeconcentracao{Healthcare}

\instituicao{\MakeUppercase{University of São Paulo}}
\universidade{University of São Paulo}

\instituicao{
  \MakeUppercase{University of São Paulo}
  \par
  \MakeUppercase{School of Nursing}
}

\escola{School of Nursing}

\instituicao{
  \MakeUppercase{University of São Paulo}
  \par
  \MakeUppercase{School of Nursing}
  \par
  \MakeUppercase{Graduate Program in Nursing}
}

\programa{Graduate Program in Nursing}

\notadeversao{\MakeTitlecase{Preliminary version}}

\hypersetup{
pdftitle={Associations Between the Duration and Quality of Sleep of Pregnant Women in the Third Trimester With the Duration of Labor},
pdfauthor={Alícia Rafaelly Vilefort Sales},
pdflang={en},
pdfsubject={Thesis},
linktoc={section},
colorlinks=true,
linkcolor={brand-primary},
filecolor={brand-primary},
citecolor={brand-primary},
urlcolor={brand-primary},
pdfcreator={LaTeX via pandoc},
bookmarksdepth=5
}
% Set sections skips (`\cftchapterpresnum`)

\setlength{\cftbeforebookskip}{0\baselineskip}
\setlength{\cftbeforepartskip}{\bigskipamount}
\setlength{\cftbeforechapterskip}{\microskipamount}
\setlength{\cftbeforesectionskip}{0\baselineskip}
\setlength{\cftbeforesubsectionskip}{0\baselineskip}
\setlength{\cftbeforesubsubsectionskip}{0\baselineskip}
\setlength{\cftbeforeparagraphskip}{0\baselineskip}

% Set section numbers fonts (`\cftchapterpresnum`)

\renewcommand{\cftchapterpresnum}{\normalfont}
\renewcommand{\cftsectionpresnum}{\normalfont}
\renewcommand{\cftsubsectionpresnum}{\normalfont}
\renewcommand{\cftsubsubsectionpresnum}{\normalfont}
\renewcommand{\cftparagraphpresnum}{\normalfont}

% Set section names fonts (`\cftpartfont`)

\renewcommand{\cftpartfont}[1]{
  \ABNTEXchapterupperifneeded{\normalfont\bfseries #1}
}

\renewcommand{\cftchapterfont}[1]{
  \ABNTEXchapterupperifneeded{\normalfont\bfseries #1}
}

\renewcommand{\cftsectionfont}[1]{
  \ABNTEXsectionupperifneeded{\normalfont #1}
}

\renewcommand{\cftsubsectionfont}[1]{
  \ABNTEXsubsectionupperifneeded{\normalfont\bfseries #1}
}

\renewcommand{\cftsubsubsectionfont}[1]{
  \ABNTEXsubsubsectionupperifneeded{\normalfont #1}
}

\renewcommand{\cftparagraphfont}[1]{
  \ABNTEXsubsubsubsectionupperifneeded{\normalfont\itshape #1}
}

% Set section page numbers fonts (`\cftpartpagefont`)

\renewcommand{\cftpartpagefont}{\normalfont}
\renewcommand{\cftchapterpagefont}{\normalfont}
\renewcommand{\cftsectionpagefont}{\normalfont}
\renewcommand{\cftsubsectionpagefont}{\normalfont}
\renewcommand{\cftsubsubsectionpagefont}{\normalfont}
\renewcommand{\cftparagraphpagefont}{\normalfont}
\renewcommand{\cftfigurepagefont}{\normalfont}
\renewcommand{\cfttablepagefont}{\normalfont}

% Renew abntex2 ToC commands -----

\cftinsertcode{A}{} % [Changed]

% This is not right. Create an issue about it.
\renewcommand{\tocprintchapternonum}{
  \addtocontents{toc}{\setlength{\cftchapterindent}{5.65em}}
  \addtocontents{toc}{\setlength{\cftchapternumwidth}{0em}}
}

\renewcommand{\tocpartapendices}{
  \addtocontents{toc}{\setlength{\cftpartindent}{5.65em}}
  \addtocontents{toc}{\setlength{\cftpartnumwidth}{0em}}
}

% Set ToC skip

\newcommand{\tocskipone}{
  \addtocontents{toc}{\protect\vspace{\smallskipamount}}
}

% \setlength{\cftbeforepartskip}{\bigskipamount}
\newcommand{\tocskiptwo}{
  % \addtocontents{toc}{\protect\vspace{\tinyskipamount}}
}
\usepackage[
style=apa
,backend=biber,language=english,url=true,useprefix=false,giveninits=true
]{biblatex}

\usepackage{csquotes}

\addbibresource{references.bib}

\renewcommand{\bibname}{REFERENCES}
\newcommand{\newbibname}{REFERENCES}

\newcommand{\bibnamewithfootnote}{
  \newbibname\protect\footnote{In accordance with the American
Psychological Association (APA) Style, 7th edition.}
}

\setlength{\bibhang}{0.5cm}

\setlength{\bibparsep}{1ex}


\defbibheading{bibheading}[\bibnamewithfootnote]{
  \ifthenelse{\boolean{ABNTEXupperchapter}}{
    \setboolean{ABNTEXupperchapter}{false}
    \chapter*{#1}
    \markboth{#1}{#1}
    \setboolean{ABNTEXupperchapter}{true}
  }{
    \chapter*{#1}
    \markboth{#1}{#1}
  }
}

\AtBeginBibliography{\vspace{0.5\baselineskip}}
\AtEveryBibitem{\clearfield{annotation}}
\renewcommand{\bibfont}{\ABNTEXfontereduzida}

% Set sections skips (`\cftchapterpresnum`)

% Credits: https://tex.stackexchange.com/a/28361/234832
% \begin{luacode}
% local PENALTY=node.id("penalty")
% last_line_twice_parindent = function (head)
%   while head do
%     local _w,_h,_d = node.dimensions(head)
%     if head.id == PENALTY and head.subtype ~= 15 and (_w < 2 * tex.parindent) then
%
%         -- we are at a glue and have less than 2*\parindent to go
%         local p = node.new("penalty")
%         p.penalty = 10000
%         p.next = head
%         head.prev.next = p
%         p.prev = head.prev
%         head.prev = p
%     end
%
%     head = head.next
%   end
%   return true
% end
%
% luatexbase.add_to_callback("pre_linebreak_filter",last_line_twice_parindent,"Raphink")
% \end{luacode}
% Use upquote if available, for straight quotes in verbatim environments
\IfFileExists{upquote.sty}{\usepackage{upquote}}{}
\IfFileExists{microtype.sty}{% use microtype if available
  \usepackage[]{microtype}
  \UseMicrotypeSet[protrusion]{basicmath} % disable protrusion for tt fonts
}{}





\setlength{\emergencystretch}{3em} % Prevent overfull lines

\setcounter{secnumdepth}{5}

% Make \paragraph and \subparagraph free-standing
\ifx\paragraph\undefined\else
  \let\oldparagraph\paragraph
  \renewcommand{\paragraph}[1]{\oldparagraph{#1}\mbox{}}
\fi
\ifx\subparagraph\undefined\else
  \let\oldsubparagraph\subparagraph
  \renewcommand{\subparagraph}[1]{\oldsubparagraph{#1}\mbox{}}
\fi


\newcolumntype{P}[1]{>{\centering\arraybackslash}p{#1}}

\clubpenalty10000
\widowpenalty10000
\displaywidowpenalty10000

\ifLuaTeX
  \usepackage{selnolig}  % disable illegal ligatures
\fi


\IfFileExists{xurl.sty}{\usepackage{xurl}}{} % add URL line breaks if available
\urlstyle{same} % disable monospaced font for URLs


% -----
% Custom functions
% -----

% Credits: <https://tex.stackexchange.com/a/300215/234832>.

\usepackage{xparse}

\ExplSyntaxOn
\NewExpandableDocumentCommand{\repeatntimes}{O{}mm}
 {
  \int_compare:nT { #2 > 0 }
   {
    #3 \prg_replicate:nn { #2 - 1 } { #1#3 }
   }
 }
\ExplSyntaxOff
%:::% class attribute begin/end %:::%

% -----
% Title page
% -----

%:::% title-page begin %:::%
\preambulo{
\hyphenpenalty=100000
%:::% title-page body begin %:::%
{\imprimirtipotrabalho} presented to the {\imprimirescola} at the {\imprimiruniversidade}, as a requirement for the degree of {\imprimirtituloacademico} by the {\imprimirprograma}.

\smallskip
Area of concentration: {\imprimirareadeconcentracao}

\smallskip
Supervisor: Prof. Dr. {\imprimirorientador}
%:::% title-page body end %:::%
}
%:::% title-page end %:::%

% -----
% Other additions
% -----

%:::% other-in-header begin %:::%
\definecolor{brand-primary}{HTML}{63483E}
\definecolor{brand-black}{HTML}{63483E}

\newcommand{\brandprimary}[1]{
  \textcolor{brand-primary}{#1}
}

\newcommand{\brandblack}[1]{
  \textcolor{brand-black}{#1}
}
%:::% other-in-header end %:::%
\makeatletter
\@ifpackageloaded{tcolorbox}{}{\usepackage[skins,breakable]{tcolorbox}}
\@ifpackageloaded{fontawesome5}{}{\usepackage{fontawesome5}}
\definecolor{quarto-callout-color}{HTML}{909090}
\definecolor{quarto-callout-note-color}{HTML}{0758E5}
\definecolor{quarto-callout-important-color}{HTML}{CC1914}
\definecolor{quarto-callout-warning-color}{HTML}{EB9113}
\definecolor{quarto-callout-tip-color}{HTML}{00A047}
\definecolor{quarto-callout-caution-color}{HTML}{FC5300}
\definecolor{quarto-callout-color-frame}{HTML}{acacac}
\definecolor{quarto-callout-note-color-frame}{HTML}{4582ec}
\definecolor{quarto-callout-important-color-frame}{HTML}{d9534f}
\definecolor{quarto-callout-warning-color-frame}{HTML}{f0ad4e}
\definecolor{quarto-callout-tip-color-frame}{HTML}{02b875}
\definecolor{quarto-callout-caution-color-frame}{HTML}{fd7e14}
\makeatother
\makeatletter
\@ifpackageloaded{bookmark}{}{\usepackage{bookmark}}
\makeatother
\makeatletter
\@ifpackageloaded{caption}{}{\usepackage{caption}}
\AtBeginDocument{%
\ifdefined\contentsname
  \renewcommand*\contentsname{Table of Contents}
\else
  \newcommand\contentsname{Table of Contents}
\fi
\ifdefined\listfigurename
  \renewcommand*\listfigurename{List of Figures}
\else
  \newcommand\listfigurename{List of Figures}
\fi
\ifdefined\listtablename
  \renewcommand*\listtablename{List of Tables}
\else
  \newcommand\listtablename{List of Tables}
\fi
\ifdefined\figurename
  \renewcommand*\figurename{Figure}
\else
  \newcommand\figurename{Figure}
\fi
\ifdefined\tablename
  \renewcommand*\tablename{Table}
\else
  \newcommand\tablename{Table}
\fi
}
\@ifpackageloaded{float}{}{\usepackage{float}}
\floatstyle{ruled}
\@ifundefined{c@chapter}{\newfloat{codelisting}{h}{lop}}{\newfloat{codelisting}{h}{lop}[chapter]}
\floatname{codelisting}{Listing}
\newcommand*\listoflistings{\listof{codelisting}{List of Listings}}
\makeatother
\makeatletter
\makeatother
\makeatletter
\@ifpackageloaded{caption}{}{\usepackage{caption}}
\@ifpackageloaded{subcaption}{}{\usepackage{subcaption}}
\makeatother
\makeatletter
\@ifpackageloaded{tcolorbox}{}{\usepackage[skins,breakable]{tcolorbox}}
\makeatother
\makeatletter
\@ifundefined{shadecolor}{\definecolor{shadecolor}{HTML}{CFD0D1}}{}
\makeatother
\makeatletter
\@ifundefined{codebgcolor}{\definecolor{codebgcolor}{HTML}{F1F3F5}}{}
\makeatother
\makeatletter
\ifdefined\Shaded\renewenvironment{Shaded}{\begin{tcolorbox}[breakable, borderline west={3pt}{0pt}{shadecolor}, boxrule=0pt, frame hidden, colback={codebgcolor}, sharp corners, enhanced]}{\end{tcolorbox}}\fi
\makeatother

% -----
% Body
% -----

\begin{document}

% Top matter -----

\pretextual

\frenchspacing

\selectlanguage{english}

%:::% class attribute begin/end %:::%

% -----
% Cover (mandatory)
% -----

%:::% cover begin %:::%
\imprimircapa
%:::% cover end %:::%

% -----
% Title page (mandatory)
% -----

%:::% approval-sheet begin %:::%
\imprimirfolhaderosto
%:::% approval-sheet end %:::%

% -----
% Cataloging record (mandatory)
% -----

%:::% cataloging-record begin %:::%
\begin{fichacatalografica}
\hyphenpenalty=100000
%:::% cataloging-record body begin %:::%
I authorize the full or partial reproduction of this work by any conventional or electronic means for the purposes of study and research, provided that the source is cited.

\vfill
\begin{center}
Cataloging in publication

Library

{\imprimirescola}

\medskip
\ABNTEXfontereduzida
\setlength{\fboxsep}{1cm}
\fbox{
\begin{minipage}[c][6cm]{12cm}
Sales, Alícia Rafaelly Vilefort

\hspace{0.5cm} {\imprimirtitulo}  / {\imprimirautor} ; supervisor, {\imprimirorientador}. -- {\imprimirdata}

\hspace{0.5cm} {\thelastpage} p : il.

\smallskip
\hspace{0.5cm} {\imprimirtipotrabalho} (\imprimirtituloacademico) -- {\imprimirprograma}, {\imprimirescola}, {\imprimiruniversidade}.

\hspace{0.5cm} {\imprimirnotadeversao}.

\smallskip
\hspace{0.5cm} 1. Pregnancy. 2. Sleep. 3. Childbirth. 4. Chronobiology. 5. Midwifery. I. Chofakian, Christiane Borges do Nascimento, super. II. Title.

\end{minipage}
}
\end{center}
\vspace{\hugeskipamount}
%:::% cataloging-record body end %:::%
\end{fichacatalografica}
%:::% cataloging-record end %:::%

% -----
% Approval sheet (mandatory)
% -----

%:::% approval-sheet begin %:::%
\begin{folhadeaprovacao}[\folhadeaprovacaoname]
\hyphenpenalty=100000
%:::% approval-sheet body begin %:::%
{\imprimirtipotrabalho} by {\imprimirautor}, under the title \textbf{\imprimirtitulo}, presented to the {\imprimirescola} at the {\imprimiruniversidade}, as a requirement for the degree of {\imprimirtituloacademico} by the {\imprimirprograma}, in the concentration area of {\imprimirareadeconcentracao}.

\vspace{\hugeskipamount}
Approved on Month, Day Year.

\vspace{\hugeskipamount}
\begin{center}
  Examination Committee
\end{center}

\vspace{\smallskipamount}
Committee Chair:

\vspace{\tinyskipamount}
\begingroup

\AtBeginEnvironment{tabular}{
  \normalsize\raggedright
  \renewcommand{\arraystretch}{2}
}

\setlength{\arrayrulewidth}{0pt}
\setlength{\tabcolsep}{0cm}
\begin{tabular}{m{2.5cm} m{13.5cm}}
  Prof. Dr. &  \\
  Institution &  \\
\end{tabular}

\vspace{\bigskipamount}
Examiners:

\vspace{\tinyskipamount}
\begin{tabular}{m{2.5cm} m{13.5cm}}
  Prof. Dr. &  \\
  Institution &  \\
  Evaluation &  \\
\end{tabular}

\vspace{\smallskipamount}
\begin{tabular}{m{2.5cm} m{13.5cm}}
  Prof. Dr. &  \\
  Institution &  \\
  Evaluation &  \\
\end{tabular}

\vspace{\smallskipamount}
\begin{tabular}{m{2.5cm} m{13.5cm}}
  Prof. Dr. &  \\
  Institution &  \\
  Evaluation &  \\
\end{tabular}
\endgroup
%:::% approval-sheet body end %:::%
\end{folhadeaprovacao}
%:::% approval-sheet end %:::%

% -----
% Acknowledgments (optional)
% -----

%:::% acknowledgments begin %:::%
\begin{agradecimentos}[\agradecimentosname]
\hyphenpenalty=100000
%:::% acknowledgments body begin %:::%

I would like to acknowledge and express my gratitude to the following
persons and organizations:

The \href{https://prip.usp.br/apoio-estudantil/}{Support Program for
Student Permanence and Education (PAPFE)} of USP, which enabled me to
get this far.

The \href{https://www.gov.br/capes/}{Coordination for the Improvement of
Higher Education Personnel (CAPES)}, for funding this work and enabling
my presence in graduate studies.

%:::% acknowledgments body end %:::%
\end{agradecimentos}
%:::% acknowledgments end %:::%

% -----
% Epigraph (optional)
% -----

%:::% epigraph begin %:::%
\begin{epigrafe}[] % \epigraphname | Keep #1 empty.
\vspace*{\fill} % Don't change it.
\begin{flushright}
%:::% epigraph body begin %:::%
\textit{Nullius in verba}\footnotemark{}

\footnotetext{
  The Royal Society. (n.d.). \textit{History of the Royal Society}. \href{https://royalsociety.org/about-us/history/}{https://royalsociety.org/about-us/history/}
}
%:::% epigraph body end %:::%
\end{flushright}
\end{epigrafe}
%:::% epigraph end %:::%

% -----
% Abstract in the vernacular language (mandatory)
% -----

%:::% vernacular-abstract begin %:::%
\begin{resumoenv}[\resumoname]
 %:::% vernacular-abstract reference begin %:::%
Sales, A. R. V. ({\imprimirdata}). \textit{\imprimirtitulo} [{\imprimirtipodetituloacademico}'s {\imprimirtipotrabalho}, {\imprimiruniversidade}].
%:::% vernacular-abstract reference end %:::%

%:::% vernacular-abstract body begin %:::%

The text below is related to the \textbf{project} of this thesis. The
final abstract can only be produced when the research is completed.

Among the biopsychosocial changes that occur during pregnancy are
alterations in the sleep-wake cycle pattern. Research suggests that
there are associations between the duration and quality of sleep in
pregnant women during the prenatal period and adverse maternal-infant
health outcomes. The primary aim of this project is to investigate the
presence/absence of significant associations between the duration and
quality of sleep in pregnant women in the third trimester and the
duration of labor. For this purpose, a study will be conducted with 133
pregnant women in the third trimester, followed up at SUS birthing
centers located in the city of São Paulo. Demographic, anthropometric,
obstetric, actigraphic, sleep-related, and psychological state data will
be collected from the participants, in addition to secondary data from
the women's medical records and prenatal books. The study has already
obtained all necessary ethical approvals from the competent authorities.
The results will be analyzed by comparing group means through an
analysis of covariance (ANCOVA). The hypothesis is that poorer sleep
quality and duration throughout pregnancy are associated with a longer
duration of labor. In addition to training and capacitating
professionals and generating knowledge on a public health issue, it is
expected that this study will promote and contribute to the development
of new services and technologies for monitoring pregnant women,
highlighting the relevance of the sleep-wake cycle for maternal-infant
health.

%:::% vernacular-abstract body end %:::%

%:::% vernacular-abstract keywords begin %:::%
\begin{tabular}{p{2.3cm} p{13.6cm}}
  \textbf{Keywords}: & Pregnancy. Childbirth. Humanizing Delivery.
  Natural Childbirth. Duration of childbirth. Obstetrics. Sleep.
  Sleep quality. Sleep Duration. Actigraphy. Chronobiology.
\end{tabular}
%:::% vernacular-abstract keywords end %:::%
\end{resumoenv}
%:::% vernacular-abstract end %:::%

% -----
% Abstract in the foreign language (mandatory)
% -----

%:::% foreign-abstract begin %:::%
\begin{resumoenv}[\resumoestrangeironame]
\begin{otherlanguage*}{brazil}
%:::% foreign-abstract reference begin %:::%
Sales, A. R. V. ({\imprimirdata}). \textit{Associações entre a duração e a qualidade do sono de gestantes no terceiro trimestre com a duração do trabalho de parto} [Dissertação de Mestrado, Universidade de São Paulo].
%:::% foreign-abstract reference end %:::%

%:::% foreign-abstract body begin %:::%

O texto abaixo está relacionado ao \textbf{projeto} desta dissertação. O
resumo final só poderá ser produzido quando a pesquisa for finalizada.

Dentre as alterações biopsicossociais que ocorrem durante a gravidez
estão as mudanças no padrão do ciclo sono-vigília. Pesquisas sugerem que
há associações entre a duração e a qualidade do sono de gestantes no
período pré-natal com desfechos adversos na saúde materno-infantil. O
objetivo deste projeto é investigar a presença/ausência de associações
significativas entre a duração e a qualidade do sono de gestantes no
terceiro trimestre com a duração do trabalho de parto. Para isso, será
realizado um estudo com 133 gestantes no terceiro trimestre gestacional
acompanhadas em casas de parto do SUS, localizadas no município de São
Paulo. Serão coletados dados sociodemográficos, antropométricos,
obstétricos, actigráficos, dados relacionados ao sono e ao estado
psicológico das participantes, além de dados secundários dos prontuários
e cadernetas das gestantes. O estudo já obteve todas as aprovações
éticas para sua operação por parte das autoridades competentes. Os
resultados serão analisados pela comparação das médias entre grupos por
meio de uma análise de covariância (ANCOVA). A hipótese é que uma menor
qualidade e duração de sono no final do terceiro trimestre gestacional
estão associadas a uma maior duração do trabalho de parto. Além de
formar e capacitar profissionais e gerar conhecimento em um assunto de
interesse público, espera-se que este estudo promova e contribua para o
desenvolvimento de novos serviços e tecnologias de acompanhamento de
gestantes, destacando a relevância do ciclo sono-vigília para a saúde
materno-infantil.

%:::% foreign-abstract body end %:::%

%:::% foreign-abstract keywords begin %:::%
\begin{tabular}{p{3.6cm} p{12.3cm}}
  \textbf{Palavras-chaves}: & Gravidez. Parto. Parto Humanizado. Parto Natural.
    Duração do parto. Obstetrícia. Sono. Qualidade do sono. Duração do sono.
    Actigrafia. Cronobiologia.
\end{tabular}
%:::% foreign-abstract keywords end %:::%
\end{otherlanguage*}
\end{resumoenv}
%:::% foreign-abstract end %:::%

% -----
% Table of contents (mandatory)
% -----

%:::% table-of-contents begin %:::%
\pdfbookmark[0]{\contentsname}{toc}
\tableofcontents*
\cleardoublepage
%:::% table-of-contents end %:::%

% -----
% Other additions
% -----

%:::% other-before-body begin %:::%
%:::% other-before-body end %:::%

% Main and back matter -----

\textual
\bookmarksetup{startatroot}

\chapter{Introduction}\label{introduction}

\begin{tcolorbox}[enhanced jigsaw, opacitybacktitle=0.6, toprule=.15mm, opacityback=0, leftrule=.75mm, breakable, title=\textcolor{quarto-callout-important-color}{\faExclamation}\hspace{0.5em}{Important}, arc=.35mm, bottomrule=.15mm, colframe=quarto-callout-important-color-frame, coltitle=black, toptitle=1mm, bottomtitle=1mm, rightrule=.15mm, colbacktitle=quarto-callout-important-color!10!white, titlerule=0mm, left=2mm, colback=white]

You are reading the work-in-progress of this thesis.

\microskip

This chapter is currently a dumping ground for ideas, and I don't
recommend reading it.

\end{tcolorbox}

You are currently viewing the preliminary print version of this master's
thesis.

This document follows the
\href{https://en.wikipedia.org/wiki/Collection_of_articles}{collection
of articles thesis} format. This first chapter serves as an introduction
to the thesis subject, providing its justification, aims, and a list of
all projects and related activities produced during its development. The
subsequent chapters consist of a series of articles connected to the
thesis, with the exception of the last one, which encompasses a
discussion and final remarks.

All analyses in this document are reproducible and were conducted using
the \href{https://www.r-project.org/}{R programming language} along with
the \href{https://quarto.org/}{Quarto} publishing system. It's worth
noting that this type of thesis is best suited for online viewing. To
access the digital version and see the latest research updates, please
visit \url{https://aliciarvilefortsales.github.io/mastersthesis/}.

Given its preliminary nature, not all chapters are ready for reading.
However, the author has chosen to display the entire state of the thesis
rather than presenting only polished sections. This approach provides
readers with a more comprehensive understanding of the work in progress.
Chapters not suitable for reading will include a call block indicating
their status.

\index{Test}

\bookmarksetup{startatroot}

\chapter{The Expression of the Sleep-Wake Cycle Throughout
Pregnancy}\label{the-expression-of-the-sleep-wake-cycle-throughout-pregnancy}

\begin{tcolorbox}[enhanced jigsaw, opacitybacktitle=0.6, toprule=.15mm, opacityback=0, leftrule=.75mm, breakable, title=\textcolor{quarto-callout-important-color}{\faExclamation}\hspace{0.5em}{Important}, arc=.35mm, bottomrule=.15mm, colframe=quarto-callout-important-color-frame, coltitle=black, toptitle=1mm, bottomtitle=1mm, rightrule=.15mm, colbacktitle=quarto-callout-important-color!10!white, titlerule=0mm, left=2mm, colback=white]

You are reading the work-in-progress of this thesis.

\microskip

This chapter is currently a dumping ground for ideas, and I don't
recommend reading it.

\end{tcolorbox}

\begin{tcolorbox}[enhanced jigsaw, opacitybacktitle=0.6, toprule=.15mm, opacityback=0, leftrule=.75mm, breakable, title=\textcolor{quarto-callout-note-color}{\faInfo}\hspace{0.5em}{Target journal}, arc=.35mm, bottomrule=.15mm, colframe=quarto-callout-note-color-frame, coltitle=black, toptitle=1mm, bottomtitle=1mm, rightrule=.15mm, colbacktitle=quarto-callout-note-color!10!white, titlerule=0mm, left=2mm, colback=white]

\begin{enumerate}
\def\labelenumi{\arabic{enumi}.}
\tightlist
\item
  \href{https://www.tandfonline.com/action/authorSubmission?show=instructions&journalCode=icbi20}{Chronobiology
  International} (\href{https://jcr.clarivate.com/jcr/}{IF 2022:
  2.8/JCR} \textbar{}
  \href{https://sucupira.capes.gov.br/sucupira/public/consultas/coleta/veiculoPublicacaoQualis/listaConsultaGeralPeriodicos.jsf}{A1/2017-2020}).
\item
  \href{https://journals.sagepub.com/author-instructions/JBR}{Journal of
  Biological Rhythms} (\href{https://jcr.clarivate.com/jcr/}{IF 2022:
  3.5/JCR} \textbar{}
  \href{https://sucupira.capes.gov.br/sucupira/public/consultas/coleta/veiculoPublicacaoQualis/listaConsultaGeralPeriodicos.jsf}{A2/2017-2020}).
\end{enumerate}

\end{tcolorbox}

\begin{tcolorbox}[enhanced jigsaw, opacitybacktitle=0.6, toprule=.15mm, opacityback=0, leftrule=.75mm, breakable, title=\textcolor{quarto-callout-note-color}{\faInfo}\hspace{0.5em}{Note}, arc=.35mm, bottomrule=.15mm, colframe=quarto-callout-note-color-frame, coltitle=black, toptitle=1mm, bottomtitle=1mm, rightrule=.15mm, colbacktitle=quarto-callout-note-color!10!white, titlerule=0mm, left=2mm, colback=white]

The following study was performed by Alícia Rafaelly Vilefort Sales
(\textbf{ARVS}), Daniel Vartanian (\textbf{DV}), Maria Augusta Medeiros
de Andrade (\textbf{MAMA}), Mario Pedrazzoli (\textbf{MP}) and
Christiane Borges do Nascimento Chofakian (\textbf{CBNC}).

\microskip

\textbf{ARVS}, \textbf{DV} and \textbf{MAMA} contributed to the study's
design. \textbf{ARVS} implemented the study, performed the statistical
analysis, and authored the manuscript. \textbf{CBNC} and \textbf{MP}
served as scientific advisors. All authors participated in discussions
about the results and contributed to the final manuscript revision.

\microskip

\emph{Future reference}: Sales, A. R. V., Vartanian, D., Andrade, M. A.
M., Pedrazzoli, M., \& Chofakian, C. B. N. (2024). The expression of the
sleep-wake cycle throughout pregnancy: a systematic and quantitative
literature review. \emph{Chronobiology International}.

\end{tcolorbox}

\bookmarksetup{startatroot}

\chapter{Associations Between the Duration and Quality of Sleep of
Pregnant Women in the Third Trimester With the Duration of
Labor}\label{associations-between-the-duration-and-quality-of-sleep-of-pregnant-women-in-the-third-trimester-with-the-duration-of-labor}

\begin{tcolorbox}[enhanced jigsaw, opacitybacktitle=0.6, toprule=.15mm, opacityback=0, leftrule=.75mm, breakable, title=\textcolor{quarto-callout-important-color}{\faExclamation}\hspace{0.5em}{Important}, arc=.35mm, bottomrule=.15mm, colframe=quarto-callout-important-color-frame, coltitle=black, toptitle=1mm, bottomtitle=1mm, rightrule=.15mm, colbacktitle=quarto-callout-important-color!10!white, titlerule=0mm, left=2mm, colback=white]

You are reading the work-in-progress of this thesis.

\microskip

This chapter is currently a dumping ground for ideas, and I don't
recommend reading it.

\end{tcolorbox}

\begin{tcolorbox}[enhanced jigsaw, opacitybacktitle=0.6, toprule=.15mm, opacityback=0, leftrule=.75mm, breakable, title=\textcolor{quarto-callout-note-color}{\faInfo}\hspace{0.5em}{Target Journal}, arc=.35mm, bottomrule=.15mm, colframe=quarto-callout-note-color-frame, coltitle=black, toptitle=1mm, bottomtitle=1mm, rightrule=.15mm, colbacktitle=quarto-callout-note-color!10!white, titlerule=0mm, left=2mm, colback=white]

\begin{enumerate}
\def\labelenumi{\arabic{enumi}.}
\tightlist
\item
  \href{https://www.tandfonline.com/action/authorSubmission?show=instructions&journalCode=icbi20}{Chronobiology
  International} (\href{https://jcr.clarivate.com/jcr/}{IF 2022:
  2.8/JCR} \textbar{}
  \href{https://sucupira.capes.gov.br/sucupira/public/consultas/coleta/veiculoPublicacaoQualis/listaConsultaGeralPeriodicos.jsf}{A1/2017-2020}).
\item
  \href{https://journals.sagepub.com/author-instructions/JBR}{Journal of
  Biological Rhythms} (\href{https://jcr.clarivate.com/jcr/}{IF 2022:
  3.5/JCR} \textbar{}
  \href{https://sucupira.capes.gov.br/sucupira/public/consultas/coleta/veiculoPublicacaoQualis/listaConsultaGeralPeriodicos.jsf}{A2/2017-2020}).
\end{enumerate}

\end{tcolorbox}

\begin{tcolorbox}[enhanced jigsaw, opacitybacktitle=0.6, toprule=.15mm, opacityback=0, leftrule=.75mm, breakable, title=\textcolor{quarto-callout-note-color}{\faInfo}\hspace{0.5em}{Note}, arc=.35mm, bottomrule=.15mm, colframe=quarto-callout-note-color-frame, coltitle=black, toptitle=1mm, bottomtitle=1mm, rightrule=.15mm, colbacktitle=quarto-callout-note-color!10!white, titlerule=0mm, left=2mm, colback=white]

The following study was performed by Alícia Rafaelly Vilefort Sales
(\textbf{ARVS}), Daniel Vartanian (\textbf{DV}), Maria Augusta Medeiros
de Andrade (\textbf{MAMA}), Mario Pedrazzoli (\textbf{MP}) and
Christiane Borges do Nascimento Chofakian (\textbf{CBNC}).

\microskip

\textbf{ARVS}, \textbf{DV} and \textbf{MAMA} contributed to the study's
design. \textbf{ARVS} implemented the study, performed the statistical
analysis, and authored the manuscript. \textbf{CBNC} and \textbf{MP}
served as scientific advisors. All authors participated in discussions
about the results and contributed to the final manuscript revision.

\microskip

\emph{Future reference}: Sales, A. R. V., Vartanian, D., Andrade, M. A.
M., Pedrazzoli, M., \& Chofakian, C. B. N. (2024). Associations between
the duration and quality of sleep of pregnant women in the third
trimester with the duration of labor. \emph{Chronobiology
International}.

\end{tcolorbox}

\bookmarksetup{startatroot}

\chapter{Conclusion}\label{conclusion}

\begin{tcolorbox}[enhanced jigsaw, opacitybacktitle=0.6, toprule=.15mm, opacityback=0, leftrule=.75mm, breakable, title=\textcolor{quarto-callout-important-color}{\faExclamation}\hspace{0.5em}{Important}, arc=.35mm, bottomrule=.15mm, colframe=quarto-callout-important-color-frame, coltitle=black, toptitle=1mm, bottomtitle=1mm, rightrule=.15mm, colbacktitle=quarto-callout-important-color!10!white, titlerule=0mm, left=2mm, colback=white]

You are reading the work-in-progress of this thesis.

\microskip

This chapter is currently a dumping ground for ideas, and I don't
recommend reading it.

\end{tcolorbox}

\postextual

\begingroup
\renewcommand{\baselinestretch}{1}
\setcounter{footnote}{0}
\renewcommand{\thefootnote}{\fnsymbol{footnote}}
\printbibliography[heading=bibheading]
\endgroup

\tocskipone
\tocprintchapternonum
\addcontentsline{toc}{chapter}{\newbibname}

\begin{glossario}

\bookmarksetup{startatroot}

\chapter*{Glossary}\label{glossary}
\addcontentsline{toc}{chapter}{Glossary}

\markboth{Glossary}{Glossary}

For an extensive list of chronobiology related terms and definitions,
please refer to \textcite{aschoff1965} and \textcite{marques2012a}.

\begin{description}
\item[Circadian rhythm]
\hspace{20cm}

A rhythm with a period close to a day/24h, an approximation to the
period of the earth's rotation \autocite{pittendrigh1960}. From the
Latin \emph{circā}, around, and \emph{dĭes}, day \autocite{latinitium}.
Example: the sleep-wake cycle.
\end{description}

\end{glossario}

\begin{apendicesenv}

\cleardoublepage
\phantomsection
\addcontentsline{toc}{part}{Appendices}
\appendix

\chapter{Appendix A}\label{appendix-a}

\begin{tcolorbox}[enhanced jigsaw, opacitybacktitle=0.6, toprule=.15mm, opacityback=0, leftrule=.75mm, breakable, title=\textcolor{quarto-callout-important-color}{\faExclamation}\hspace{0.5em}{Important}, arc=.35mm, bottomrule=.15mm, colframe=quarto-callout-important-color-frame, coltitle=black, toptitle=1mm, bottomtitle=1mm, rightrule=.15mm, colbacktitle=quarto-callout-important-color!10!white, titlerule=0mm, left=2mm, colback=white]

You are reading the work-in-progress of this thesis.

\microskip

This chapter is currently a dumping ground for ideas, and I don't
recommend reading it.

\end{tcolorbox}

\end{apendicesenv}

% -----
% Other additions
% -----

%:::% other-after-body begin %:::%
%:::% other-after-body end %:::%

\end{document}
