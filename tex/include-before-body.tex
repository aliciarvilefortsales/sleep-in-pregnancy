%:::% class attribute begin/end %:::%

% -----
% Cover (mandatory)
% -----

%:::% cover begin %:::%
\imprimircapa
%:::% cover end %:::%

% -----
% Title page (mandatory)
% -----

%:::% approval-sheet begin %:::%
\imprimirfolhaderosto
%:::% approval-sheet end %:::%

% -----
% Cataloging record (mandatory)
% -----

%:::% cataloging-record begin %:::%
\begin{fichacatalografica}
\hyphenpenalty=100000
%:::% cataloging-record body begin %:::%
I authorize the full or partial reproduction of this work by any conventional or electronic means for the purposes of study and research, provided that the source is cited.

\vfill
\begin{center}
Cataloging in publication

Library

{\imprimirescola}

\medskip
\ABNTEXfontereduzida
\setlength{\fboxsep}{1cm}
\fbox{
\begin{minipage}[c][6cm]{12cm}
Sales, Alícia Rafaelly Vilefort

\hspace{0.5cm} {\imprimirtitulo}  / {\imprimirautor} ; supervisor, {\imprimirorientador}. -- {\imprimirdata}

\hspace{0.5cm} {\thelastpage} p : il.

\smallskip
\hspace{0.5cm} {\imprimirtipotrabalho} (\imprimirtituloacademico) -- {\imprimirprograma}, {\imprimirescola}, {\imprimiruniversidade}.

\hspace{0.5cm} {\imprimirnotadeversao}.

\smallskip
\hspace{0.5cm} 1. Pregnancy. 2. Sleep. 3. Childbirth. 4. Chronobiology. 5. Midwifery. I. Chofakian, Christiane Borges do Nascimento, super. II. Title.

\end{minipage}
}
\end{center}
\vspace{\hugeskipamount}
%:::% cataloging-record body end %:::%
\end{fichacatalografica}
%:::% cataloging-record end %:::%

% -----
% Approval sheet (mandatory)
% -----

%:::% approval-sheet begin %:::%
\begin{folhadeaprovacao}[\folhadeaprovacaoname]
\hyphenpenalty=100000
%:::% approval-sheet body begin %:::%
{\imprimirtipotrabalho} by {\imprimirautor}, under the title \textbf{\imprimirtitulo}, presented to the {\imprimirescola} at the {\imprimiruniversidade}, as a requirement for the degree of {\imprimirtituloacademico} by the {\imprimirprograma}, in the concentration area of {\imprimirareadeconcentracao}.

\vspace{\hugeskipamount}
Approved on Month, Day Year.

\vspace{\hugeskipamount}
\begin{center}
  Examination Committee
\end{center}

\vspace{\smallskipamount}
Committee Chair:

\vspace{\tinyskipamount}
\begingroup

\AtBeginEnvironment{tabular}{
  \normalsize\raggedright
  \renewcommand{\arraystretch}{2}
}

\setlength{\arrayrulewidth}{0pt}
\setlength{\tabcolsep}{0cm}
\begin{tabular}{m{2.5cm} m{13.5cm}}
  Prof. Dr. &  \\
  Institution &  \\
\end{tabular}

\vspace{\bigskipamount}
Examiners:

\vspace{\tinyskipamount}
\begin{tabular}{m{2.5cm} m{13.5cm}}
  Prof. Dr. &  \\
  Institution &  \\
  Evaluation &  \\
\end{tabular}

\vspace{\smallskipamount}
\begin{tabular}{m{2.5cm} m{13.5cm}}
  Prof. Dr. &  \\
  Institution &  \\
  Evaluation &  \\
\end{tabular}

\vspace{\smallskipamount}
\begin{tabular}{m{2.5cm} m{13.5cm}}
  Prof. Dr. &  \\
  Institution &  \\
  Evaluation &  \\
\end{tabular}
\endgroup
%:::% approval-sheet body end %:::%
\end{folhadeaprovacao}
%:::% approval-sheet end %:::%

% -----
% Acknowledgments (optional)
% -----

%:::% acknowledgments begin %:::%
\begin{agradecimentos}[\agradecimentosname]
\hyphenpenalty=100000
%:::% acknowledgments body begin %:::%

I would like to acknowledge and express my gratitude to the following
persons and organizations:

The \href{https://prip.usp.br/apoio-estudantil/}{Support Program for
Student Permanence and Education (PAPFE)} of USP, which enabled me to
get this far.

The \href{https://www.gov.br/capes/}{Coordination for the Improvement of
Higher Education Personnel (CAPES)}, for funding this work and enabling
my presence in graduate studies.

%:::% acknowledgments body end %:::%
\end{agradecimentos}
%:::% acknowledgments end %:::%

% -----
% Epigraph (optional)
% -----

%:::% epigraph begin %:::%
\begin{epigrafe}[] % \epigraphname | Keep #1 empty.
\vspace*{\fill} % Don't change it.
\begin{flushright}
%:::% epigraph body begin %:::%
\textit{Nullius in verba}\footnotemark{}

\footnotetext{
  The Royal Society. (n.d.). \textit{History of the Royal Society}. \href{https://royalsociety.org/about-us/history/}{https://royalsociety.org/about-us/history/}
}
%:::% epigraph body end %:::%
\end{flushright}
\end{epigrafe}
%:::% epigraph end %:::%

% -----
% Abstract in the vernacular language (mandatory)
% -----

%:::% vernacular-abstract begin %:::%
\begin{resumoenv}[\resumoname]
 %:::% vernacular-abstract reference begin %:::%
Sales, A. R. V. ({\imprimirdata}). \textit{\imprimirtitulo} [{\imprimirtipodetituloacademico}'s {\imprimirtipotrabalho}, {\imprimiruniversidade}].
%:::% vernacular-abstract reference end %:::%

%:::% vernacular-abstract body begin %:::%

The text below is related to the \textbf{project} of this thesis. The
final abstract can only be produced when the research is completed.

Among the biopsychosocial changes that occur during pregnancy are
alterations in the sleep-wake cycle pattern. Research suggests that
there are associations between the duration and quality of sleep in
pregnant women during the prenatal period and adverse maternal-infant
health outcomes. The primary aim of this project is to investigate the
presence/absence of significant associations between the duration and
quality of sleep in pregnant women in the third trimester and the
duration of labor. For this purpose, a study will be conducted with 133
pregnant women in the third trimester, followed up at SUS birthing
centers located in the city of São Paulo. Demographic, anthropometric,
obstetric, actigraphic, sleep-related, and psychological state data will
be collected from the participants, in addition to secondary data from
the women's medical records and prenatal books. The study has already
obtained all necessary ethical approvals from the competent authorities.
The results will be analyzed by comparing group means through an
analysis of covariance (ANCOVA). The hypothesis is that poorer sleep
quality and duration throughout pregnancy are associated with a longer
duration of labor. In addition to training and capacitating
professionals and generating knowledge on a public health issue, it is
expected that this study will promote and contribute to the development
of new services and technologies for monitoring pregnant women,
highlighting the relevance of the sleep-wake cycle for maternal-infant
health.

%:::% vernacular-abstract body end %:::%

%:::% vernacular-abstract keywords begin %:::%
\begin{tabular}{p{2.3cm} p{13.6cm}}
  \textbf{Keywords}: & Pregnancy. Childbirth. Humanizing Delivery.
  Natural Childbirth. Duration of childbirth. Obstetrics. Sleep.
  Sleep quality. Sleep Duration. Actigraphy. Chronobiology.
\end{tabular}
%:::% vernacular-abstract keywords end %:::%
\end{resumoenv}
%:::% vernacular-abstract end %:::%

% -----
% Abstract in the foreign language (mandatory)
% -----

%:::% foreign-abstract begin %:::%
\begin{resumoenv}[\resumoestrangeironame]
\begin{otherlanguage*}{brazil}
%:::% foreign-abstract reference begin %:::%
Sales, A. R. V. ({\imprimirdata}). \textit{Associações entre a duração e a qualidade do sono de gestantes no terceiro trimestre com a duração do trabalho de parto} [Dissertação de Mestrado, Universidade de São Paulo].
%:::% foreign-abstract reference end %:::%

%:::% foreign-abstract body begin %:::%

O texto abaixo está relacionado ao \textbf{projeto} desta dissertação. O
resumo final só poderá ser produzido quando a pesquisa for finalizada.

Dentre as alterações biopsicossociais que ocorrem durante a gravidez
estão as mudanças no padrão do ciclo sono-vigília. Pesquisas sugerem que
há associações entre a duração e a qualidade do sono de gestantes no
período pré-natal com desfechos adversos na saúde materno-infantil. O
objetivo deste projeto é investigar a presença/ausência de associações
significativas entre a duração e a qualidade do sono de gestantes no
terceiro trimestre com a duração do trabalho de parto. Para isso, será
realizado um estudo com 133 gestantes no terceiro trimestre gestacional
acompanhadas em casas de parto do SUS, localizadas no município de São
Paulo. Serão coletados dados sociodemográficos, antropométricos,
obstétricos, actigráficos, dados relacionados ao sono e ao estado
psicológico das participantes, além de dados secundários dos prontuários
e cadernetas das gestantes. O estudo já obteve todas as aprovações
éticas para sua operação por parte das autoridades competentes. Os
resultados serão analisados pela comparação das médias entre grupos por
meio de uma análise de covariância (ANCOVA). A hipótese é que uma menor
qualidade e duração de sono no final do terceiro trimestre gestacional
estão associadas a uma maior duração do trabalho de parto. Além de
formar e capacitar profissionais e gerar conhecimento em um assunto de
interesse público, espera-se que este estudo promova e contribua para o
desenvolvimento de novos serviços e tecnologias de acompanhamento de
gestantes, destacando a relevância do ciclo sono-vigília para a saúde
materno-infantil.

%:::% foreign-abstract body end %:::%

%:::% foreign-abstract keywords begin %:::%
\begin{tabular}{p{3.6cm} p{12.3cm}}
  \textbf{Palavras-chaves}: & Gravidez. Parto. Parto Humanizado. Parto Natural.
    Duração do parto. Obstetrícia. Sono. Qualidade do sono. Duração do sono.
    Actigrafia. Cronobiologia.
\end{tabular}
%:::% foreign-abstract keywords end %:::%
\end{otherlanguage*}
\end{resumoenv}
%:::% foreign-abstract end %:::%

% -----
% Table of contents (mandatory)
% -----

%:::% table-of-contents begin %:::%
\pdfbookmark[0]{\contentsname}{toc}
\tableofcontents*
\cleardoublepage
%:::% table-of-contents end %:::%

% -----
% Other additions
% -----

%:::% other-before-body begin %:::%
%:::% other-before-body end %:::%
